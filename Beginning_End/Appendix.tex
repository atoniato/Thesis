\appendix

\chapter{}

\section{General conventions and definitions}

In this appendix we introduce the general notations and conventions used in the thesis. 

\subsection{Metric and units}

We use the following convention for Minkowski metric:

\begin{equation}
(\eta_{\mu\nu}) = \mathrm{diag}[1,-1,-1,-1] \: .
\label{metric}
\end{equation}

As for the unit system, we use natural units, where the speed of light $c$, the reduced Planck constant $\hbar$ and the Boltzmann constant $k_B$ are all set to unity.

\subsection{Gamma matrices}
\label{gamma_matrices}

Gamma matrices appear in the Dirac Lagrangian: $\Lag_D [\bar \psi, \psi] = \bar \psi (i \gamma^{\mu} D_{\mu} -m \id) \psi$, and fulfil the following relations:

\begin{itemize}
\item Clifford algebra: $\qty \big{\gamma^{\mu}, \gamma^{\nu}} = 2 \id \eta^{\mu \nu}$
\item Relation with the hermitian conjugate: $\gamma^{\mu} = \gamma^0 (\gamma^{\mu})^{\dagger} \gamma^0$ \: .
\end{itemize}
%
As a consequence, the following equations hold:
\begin{itemize}
\item $(\gamma^0)^2 = \id$, $(\gamma^i)^2 = - \id$, $i= 1,2,3$
\item $(\gamma^0)^{\dagger} = \gamma^0$, $(\gamma^i)^{\dagger} = -\gamma^i$, $i= 1,2,3$
\item $\qty \big{\gamma^{\mu}, \gamma^{\nu}} = 0$ if $\mu \neq \nu$ \: .
\end{itemize}

$\gamma_5$ is defined by: $\gamma_5 = i \gamma^0 \gamma^1 \gamma^2 \gamma^3$, and it has the following properties: $(\gamma_5)^2=\id$, $(\gamma_5)^{\dagger} = \gamma_5$, $\qty \big{\gamma_5, \gamma^{\mu}} = 0$. There exist multiple representations of the gamma matrices, and physical results do not depend on the choice of the representation. In this thesis we use the chiral representation:

\begin{equation}
\gamma^0 = 
\begin{pmatrix}
0 & \id_{2 \times 2} \\
\id_{2 \times 2} & 0
\end{pmatrix}
\: ,
\qquad
\gamma^i = 
\begin{pmatrix}
0 & \sigma^i \\
-\sigma^i & 0
\end{pmatrix}
\: ,
\qquad
\gamma_5 = 
\begin{pmatrix}
-\id_{2 \times 2} & 0 \\
0  & \id_{2 \times 2}
\end{pmatrix}
\: ,
\end{equation}
%
where $\sigma^i$, $i=1,2,3$, are the Pauli matrices.

Gamma matrices are involved in the definition of the Lorentz group representation acting on Dirac spinors. In particular, the generators of Lorentz transformations are defined by: $J^{\mu \nu} = -i/4 [\gamma^{\mu},\gamma^{\nu}]$. Moreover, the parity transformation of a Dirac spinor is defined by:

\begin{equation}
\psi(\textbf x, t) \to \psi'(\textbf x, t) = \gamma^0 \psi(-\textbf x, t) \: ,
\end{equation}
%
and the transformation under charge conjugation by:

\begin{equation}
\psi(x) \to \psi^C(x) = C \bar \psi(x)^T \: ,
\end{equation}
%
where the charge conjugation operator is $C = i \gamma^2 \gamma^0$.

$\gamma_5$ appears in the definition of the chirality projectors:

\begin{equation}
P_L = \frac{\id - \gamma_5}{2} \: , \qquad P_R = \frac{\id + \gamma_5}{2} \: ,
\end{equation}
%
which have the following properties: $P_{R,L}^2 = P_{R,L}$, $P_R P_L = P_L P_R = 0$, $P_L + P_R = \id$, and are used to define left- and right-handed spinors:

\begin{equation}
\psi(x) = \psi_L(x)+ \psi_R(x) \: , \qquad\psi_L(x) = P_L \psi(x) \: , \qquad \psi_R(x) = P_R \psi(x) \: .
\end{equation}
%
The Dirac Lagrangian can be rewritten in terms of left- and right-handed Dirac spinors as follows:

\begin{equation}
\Lag_D = \bar \psi_L (i \gamma^{\mu} D_{\mu}) \psi_L + \bar \psi_R (i \gamma^{\mu} D_{\mu}) \psi_R -m (\bar\psi_L \psi_R + \bar\psi_R \psi_L) \: .
\end{equation}

The Dirac Lagrangian in Euclidean space is: $\Lag_D^E [\bar \psi, \psi] = \bar \psi (\gamma_{\mu}^E D_{\mu} + m \id) \psi$, where the relation between Minkowskian and Euclidean gamma matrices is:

\begin{equation}
\gamma_0^E = \gamma_0 \: , \qquad \gamma_i^E = i \gamma_i = -i \gamma^i \: .
\end{equation}
%
Euclidean gamma matrices have the following properties: 

\begin{itemize}
\item $\qty \big{\gamma_{\mu}^E, \gamma_{\nu}^E} = 2 \id \delta_{\mu \nu} \Rightarrow (\gamma_{\mu}^E)^2 = \id$
\item $(\gamma_{\mu}^E)^{\dagger} = \gamma_{\mu}^E$
\end{itemize}

Euclidean $\gamma_5$ is defined by: $\gamma_5^E =  \gamma_1^E \gamma_2^E \gamma_3^E \gamma_0^E$, and has analogue properties to Minkowskian $\gamma_5$: $(\gamma_5^E)^2=\id$, $(\gamma_5^E)^{\dagger} = \gamma_5^E$, $\qty \big{\gamma_5^E, \gamma_{\mu}^E} = 0$.

Euclidean gamma matrices in the chiral representation are given by:

\begin{equation}
\gamma_0^E = 
\begin{pmatrix}
0 & \id_{2 \times 2} \\
\id_{2 \times 2} & 0
\end{pmatrix}
\: ,
\qquad
\gamma_i^E = 
\begin{pmatrix}
0 & -i \sigma^i \\
i \sigma^i & 0
\end{pmatrix}
\: ,
\qquad
\gamma_5^E = 
\begin{pmatrix}
\id_{2 \times 2} & 0 \\
0  & -\id_{2 \times 2}
\end{pmatrix}
\: .
\end{equation}

In this thesis we always omit the superscript $E$ in gamma matrices, and we implicitly assume that every time we discuss a lattice model, which requires the Euclidean version of the action, gamma matrices assume their Euclidean form.

\subsection{SU($N$) generators}
\label{SUN_generators}

We denote the generators of the representation $r$ of SU($N$) by $T^a_r$, with $a = 1, \dots, N^2 -1$. The generators are normalised as:

\begin{equation}
\tr [T^a_r T^b_r] = T[r] \delta^{ab} \: .
\end{equation}
%
The quadratic Casimir $C_2[r]$ is defined by:

\begin{equation}
\sum_{a=1}^{N^2 -1} T^a_r T^a_r = C_2[r] \id \: ,
\end{equation}
%
and it is related to $T[r]$ and the dimension of the representation $d[r]$ by:

\begin{equation}
C_2[r] d[r] = T[r] d[G] \: ,
\end{equation}
%
where $G$ denotes the adjoint representation. The values of $d$, $T$ and $C_2$ for the fundamental and adjoint representation are listed in table \ref{group_factors}.

    \begin{table}
\begin{center}
    %\begin{minipage}{3.8in}
    \begin{tabular}{c||ccc }
    $r$ & $ \quad T[r] $ & $\quad C_2[r] $ & $\quad d[r] $  \\
    \hline \hline
    $ \fund $ & $\quad \frac{1}{2}$ & $\quad\frac{N^2-1}{2N}$ &\quad $N$  \\
        $\text{$G$}$ &\quad $N$ &\quad $N$ &\quad $N^2-1$  \\
    \end{tabular}
    %\end{minipage}
    \end{center}
\caption{Values of $d$, $T$ and $C_2$ for the fundamental ($\fund$) and adjoint (G) representation of SU($N$).}
\label{group_factors}
    \end{table}



\section{Lattice Fourier transforms}
\label{Fourier}

In this appendix we define Fourier transforms on the lattice. The lattice $\Lambda$ is defined as: 

\begin{equation}
\Lambda = \set{x_{\mu} = n_{\mu} a  \mid  n_{\mu}  = 0, \dots, L_{\mu} -1, \; \mu = 0, \dots, 3 } \; ,
\end{equation}
%
where $a$ is the lattice spacing and $L_{ \mu}$ the number of lattice points in direction $\mu$. The total number of lattice points $V$ is given by: $V = \prod_{\mu=0}^3 L_{\mu}$.

We consider a function $f(x)$ defined on the lattice, and we define its Fourier transform as:

\begin{equation}
\begin{split}
&f(x) = \frac{1}{\sqrt V} \sum_{p \in \tilde\Lambda} \tilde f(p) e^{i p \cdot x}\\
&\tilde f(p) = \frac{1}{\sqrt V} \sum_{x \in \Lambda}  f(x) e^{-i p \cdot x}
\end{split}
\end{equation}
%
where $\tilde\Lambda$ denotes the momentum space. Since the space-time has a finite volume, the momenta belong to a discrete set. Specifically, if periodic boundary conditions are imposed in direction $\mu$, i.e. $f(x+aL_{\mu} \hat\mu) = f(x)$, then $p_{\mu}$ takes values in the discrete set $\set{p_{\mu} = \frac{2\pi}{a L_{\mu}} k_{\mu} \mid k_{\mu} \in \mathbb{Z}}$. While in the case of anti-periodic boundary conditions, $f(x+aL_{\mu} \hat\mu) = -f(x)$, $p_{\mu}$ belongs to $\set{p_{\mu} = \frac{2\pi}{a L_{\mu}} (\frac{1}{2} + k_{\mu}) \mid k_{\mu} \in \mathbb{Z}}$. Moreover, since the coordinates $x_{\mu}$ of each lattice point are represented by integer numbers of lattice spacings, the momenta are periodic: $\tilde f(p) = \tilde f(p + \frac{2\pi}{a}\hat\mu)$. Therefore, the momentum space is defined as:

\begin{equation}
\tilde\Lambda = \set{p_{\mu} = \frac{2 \pi}{a L_{\mu}}(k_{\mu} + \theta_{\mu})  \mid  k_{\mu} = -\frac{L_{\mu}}{2} + 1, \dots, \frac{L_{\mu}}{2}, \; \mu = 0, \dots, 3 } \: ,
\end{equation}
%
where $\theta_{\mu} = 0$ corresponds to periodic boundary conditions in direction $\mu$, and $\theta_{\mu} = 1/2$ to anti-periodic ones.

The lattice delta functions are defined as:

\begin{equation}
\begin{split}
& \delta(x-x') =  \delta_{n_0,n'_0} \delta_{n_1,n'_1} \delta_{n_2,n'_2} \delta_{n_3,n'_3} = \frac{1}{V} \sum_{p \in \tilde\Lambda}
e^{i p \cdot (x - x')} \\
& \delta(p-p') =  \delta_{k_0,k'_0}\delta_{k_1,k'_1} \delta_{k_2,k'_2} \delta_{k_3,k'_3} = \frac{1}{V} \sum_{x \in \Lambda} 
e^{i (p - p') \cdot x } \: .
\end{split}
\end{equation}

\section{Hybrid Monte Carlo and the detailed balance relation}
\label{HMC_db}

In this appendix we show that the HMC algorithm respects the detailed balance relation \ref{detailed_bal}. We consider a scalar field theory with action $S[\phi]$. According to equation \ref{O_HMC}, the expectation value of an observable $O$ is given by:

\begin{equation}
\langle O \rangle = \frac{\int \prod_{x} \mathrm{d} \phi(x) \mathrm{d} \pi(x) \: O[\phi] e^{-H[\pi,\phi] }}{\int \prod_{x} \mathrm{d} \phi(x) \mathrm{d} \pi(x) \: e^{-H[\pi,\phi]}} \: ,
\end{equation}
%
where the Hamiltonian is defined by: $H[\pi,\phi] =  \frac{1}{2} \sum_x \pi(x)^2 + S[\phi]$.

 The HMC algorithm consists of the following steps:

\begin{itemize}
\item The momenta $\pi(x)$ are extracted from a Gaussian distribution, with probability $P[\pi(x)] \propto \exp[-\pi(x)^2/2]$. This way an initial configuration of the momenta is defined
\item Starting from the initial configuration $(\pi, \phi)$, the system evolves to a new state $(\pi',\phi')$ along a molecular dynamics trajectory which is a numerical solution of Hamilton's equations \ref{Hamilton_eq}
\item The final configuration $(\pi',\phi')$ is accepted or rejected in a Metropolis step, according to the transition probability 
\begin{equation}
W_M(\pi, \phi \to \pi' \phi') = \min \biggl[ 1, \frac{e^{-H[\pi',\phi']}}{e^{-H[\pi,\phi]}} \biggr] \: .
\end{equation}
\end{itemize}

We denote by $W_{\mathrm{md}}(\pi,\phi \to \pi',\phi')$ the probability of evolving from the state $(\pi,\phi)$ to $(\pi',\phi')$ along the molecular dynamics trajectory.  We stress however that molecular dynamics is a deterministic process, and, given an initial state, all the following states in the trajectory are uniquely determined.

We further assume that the numerical integrator generating the molecular dynamics trajectory fulfils the following conditions:

\begin{itemize}
\item Preservation of the integration measure: $ \prod_{x} \mathrm{d} \phi(x) \mathrm{d} \pi(x) =  \prod_{x} \mathrm{d} \phi'(x) \mathrm{d} \pi'(x)$
\item Reversibility: $W_{\mathrm{md}}(\pi,\phi \to \pi',\phi') = W_{\mathrm{md}}(-\pi',\phi' \to -\pi,\phi )$\: .
\end{itemize}


The probability to evolve from a scalar field configuration $\phi$ to $\phi'$ in one HMC step is given by:

\begin{equation}
W(\phi \to \phi') = \int \prod_{x} \mathrm{d} \pi(x) \mathrm{d} \pi'(x)  W_M(\pi,\phi \to \pi',\phi') W_{\mathrm{md}}(\pi,\phi \to \pi',\phi') e^{-\sum_x \pi(x)^2 /2} \: .
\label{proofP}
\end{equation}
%
We rewrite Metropolis transition probability as follows:

\begin{equation}
\begin{split}
W_M(\pi,\phi \to \pi',\phi') & = \min \biggl[ 1, \frac{e^{-H[\pi',\phi']}}{e^{-H[\pi,\phi]}} \biggr] = 
 \frac{e^{-H[\pi',\phi']}}{e^{-H[\pi,\phi]}} \min \biggl[ 1, \frac{e^{-H[\pi,\phi]}}{e^{-H[\pi',\phi']}} \biggr]  = \\
& = \exp \biggl[ - \frac{1}{2} \sum_x \pi'(x)^2 - S[\phi'] +  \frac{1}{2} \sum_x \pi(x)^2 + S[\phi] \biggr]  W_M(\pi',\phi' \to \pi,\phi) \: ,
\end{split}
\end{equation}
%
and we insert this expression in equation \ref{proofP}:

\begin{equation}
\begin{split}
W(\phi \to \phi') = & \int \prod_{x} \mathrm{d} \pi(x) \mathrm{d} \pi'(x) W_M(\pi',\phi' \to \pi,\phi)  W_{\mathrm{md}}(\pi,\phi \to \pi',\phi') \times \\
&  \exp \biggl[- \frac{1}{2} \sum_x \pi'(x)^2 - S[\phi'] + S[\phi] \biggr]  \:  .  
\end{split}
\end{equation}

We now use the  reversibility of the molecular dynamics trajectory, together with the fact that $W_M(\pi,\phi \to \pi',\phi')$ is quadratic in $\pi$ and $\pi'$:

\begin{equation}
\begin{split}
W(\phi \to \phi') = & \exp \bigl[ - S[\phi'] + S[\phi] \bigr] \int \prod_{x} \mathrm{d} \pi(x) \mathrm{d} \pi'(x) W_M(-\pi',\phi' \to -\pi,\phi) \times \\
& W_{\mathrm{md}}(-\pi',\phi' \to -\pi,\phi) \exp \bigl[ -\sum_x \pi'(x)^2 /2 \bigr] = \\
= &  \exp \bigl[ - S[\phi'] + S[\phi] \bigr] W(\phi' \to \phi) \: ,
\end{split}
\label{proofP2}
\end{equation}
%
where in the last step we used the fact that the integration measure $ \int \prod_{x} \mathrm{d} \pi(x) \mathrm{d} \pi'(x)$ and the factor $\exp \bigl[ -\sum_x \pi'(x)^2 /2 \bigr]$ are invariant under a change of sign of $\pi$ and $\pi'$. The detailed balance relation follows directly from equation \ref{proofP2}:

\begin{equation}
\exp \bigl[ -  S[\phi] \bigr]W(\phi \to \phi') =\exp \bigl[ - S[\phi'] \bigr] W(\phi' \to \phi) \: .
\end{equation}

In the above proof we used explicitly the reversibility of the molecular dynamics process. The preservation of the integration measure $ \prod_{x} \mathrm{d} \phi(x) \mathrm{d} \pi(x) =  \prod_{x} \mathrm{d} \phi'(x) \mathrm{d} \pi'(x)$ is also required, because it ensures that in the path integral  $\int \prod_{x} \mathrm{d} \phi(x) \mathrm{d} \pi(x) \: e^{-H[\pi,\phi]}$ each configuration is actually weighted by $e^{-H[\pi,\phi]}$, without the appearance of an extra Jacobian determinant. 


\section{Scalar quartic potential in an SU(2) gauge theory}
\label{app_quartic_potential}

We consider a theory of gauge-interacting scalars in the fundamental representation of the gauge group SU(2). We discuss which terms are allowed by colour and flavour symmetries in the quartic potential. In the case of a theory with a single scalar field, we also discuss the possibility of spontaneous flavour symmetry breaking due to a scalar condensate.

\subsection{Single scalar field}


We first consider the case of one single scalar field $\phi$. A gauge transformation is given by:
%
\begin{equation}
\phi \to \phi' = U \phi
\end{equation}
%
where $U \in$ SU(2)=Sp(2) is characterised by: 
%
\begin{equation}
\begin{cases}
U^{\dagger}U = \id \\
U^{T}\epsilon U = \epsilon
\end{cases} \: ,
\end{equation}
%
$\epsilon$ being the two-dimensional antisymmetric tensor:
%
\begin{equation}
\epsilon =
\begin{pmatrix}
0 & 1 \\
-1 & 0
\end{pmatrix} \, .
\end{equation}
%
We can construct a field $\tilde{\phi}=-\epsilon \phi^*$ which transforms in the same way as $\phi$ under a gauge transformation:  $\tilde{\phi} \to \tilde{\phi}' = U \tilde{\phi}$.
%
We define a matrix $S$

\begin{equation}
S = (\phi, \tilde{\phi})
\label{S1}
\end{equation}
%
which transforms as: $S \to S'= US$.
The relation $\tilde{\phi}=-\epsilon \phi^*$ is translated in a relation between $S$ and $S^*$:

\begin{equation}
S^* = - \epsilon S E \, ,
\label{constraint}
\end{equation}
%
where
%
\begin{equation}
E =
\begin{pmatrix}
0 & 1 \\
-1 & 0
\end{pmatrix} \, .
\end{equation}
%
Eq.~\ref{constraint} can be verified as follows: we write the entries of $S$ as $S_{ia}$, where $i = 1,2$ is the colour index and $a = 1,2$ is the flavour index, then we have $S_{i1} = \phi_i$, $S_{i2} = - \epsilon_{ij} \phi^*_j = - \epsilon_{ij} S^*_{j1}$. It follows that:

\begin{equation}
\begin{split}
& (\epsilon S E)_{i1} = \epsilon_{ij} S_{ja} E_{a1} = - \epsilon_{ij} S_{j2} = -\epsilon_{ij} (- \epsilon_{jk} S_{k1}^*) = -\delta_{ik}S_{k1}^* = -S_{i1}^* \\
& (\epsilon S E)_{i2} = \epsilon_{ij} S_{ja} E_{a2} = \epsilon_{ij} S_{j1} = - S_{i2}^*
\end{split} \: .
\end{equation}
%
The kinetic term of the scalar Lagrangian is expressed in terms of $S$ as follows:

\begin{equation}
\mathcal{L}_{kin} = \frac{1}{2} \mathrm{Tr} \bigl[ (D_{\mu}S)^{\dagger} (D^{\mu}S) \bigr] \, .
\label{kin}
\end{equation}
%
Given a matrix $M \in$ U(2) we define a flavour transformation

\begin{equation}
S \to S' = SM \: ,
\end{equation}
%
and we work out the conditions under which it is a symmetry of the kinetic term \ref{kin}. If we require $S'^* = - \epsilon S' E$, it follows that $M$ must fulfil the following constraint:

\begin{equation}
EM^* = ME \: ,
\end{equation}
%
which, together with the fact that M is a unitary matrix, implies:

\begin{equation}
E=MEM^T \, .
\end{equation}
%
The flavour symmetry group is therefore Sp(2)=SU(2).

The possible terms which are quartic in the field $\phi$ and symmetric under colour and flavour transformations are the following:

\begin{equation}
\begin{split}
& \bigl( \mathrm{Tr} \bigl[ S^{\dagger} S \bigr] \bigr)^2 \\
& \mathrm{Tr} \bigl[ S^{\dagger} S S^{\dagger} S\bigr] \\
&  \bigl( \mathrm{Tr} \bigl[ S^T \epsilon S E \bigr] \bigr)^2 \\
& \mathrm{Tr} \bigl[ S^T \epsilon S E S^T \epsilon S E\bigr] \\
\end{split} \: .
\label{quartic}
\end{equation}
%
The last two terms in \ref{quartic} can be shown to be equal to the first ones by applying equation \ref{constraint}:
 
\begin{equation}
\begin{split}
\bigl( \mathrm{Tr} \bigl[ S^T \epsilon S E \bigr] \bigr)^2  &=   \bigl( \mathrm{Tr} \bigl[ S^T (-S^*) \bigr] \bigr)^2 =  \bigl( - \mathrm{Tr} \bigl[ (S^{\dagger} S)^* \bigr] \bigr)^2 \\
 & = \bigl(  \mathrm{Tr} \bigl[ (S^{\dagger} S) \bigr] \bigr)^2 
 \end{split} \: ,
 \label{proof1}
\end{equation}

\begin{equation}
\begin{split}
\mathrm{Tr} \bigl[ S^T \epsilon S E S^T \epsilon S E\bigr]  & = \mathrm{Tr} \bigl[ S^T (-S^*) S^T (-S^*) \bigr] = \mathrm{Tr} \bigl[ (S^{\dagger} S)^* (S^{\dagger} S)^*) \bigr] \\
& =\mathrm{Tr} \bigl[ S^{\dagger} S S^{\dagger} S\bigr] \, .
\end{split}
\label{proof2}
\end{equation}
%
The first and the second term in equation \ref{quartic} are not linearly independent when only one scalar field is present. To show this we start by explicitly writing $S^{\dagger} S$:

\begin{equation}
\begin{split}
& S^{\dagger} S = 
\begin{pmatrix}
\phi^{\dagger}\phi  & \phi^{\dagger} \tilde{\phi} \\
\tilde{\phi}^{\dagger} \phi & \tilde{\phi}^{\dagger} \tilde{\phi} \\
\end{pmatrix} \\
& \phi^{\dagger} \tilde{\phi} = - \phi^*_i \epsilon_{ij} \phi^*_j = 0 \\
& \tilde{\phi}^{\dagger} \phi = - \epsilon_{ij} \phi_j\phi_i = 0 \\
& \tilde{\phi}^{\dagger} \tilde{\phi} = - \epsilon_{ij} \phi_j (- \epsilon_{ik} \phi^*_k) = \delta_{jk} \phi_j \phi^*_k = \phi^{\dagger} \phi \\
\end{split}
\quad \Rightarrow  S^{\dagger} S = \phi^{\dagger} \phi \, \id \: .
\label{app_S1}
\end{equation}
%
It follows that: $(\mathrm{Tr} [S^{\dagger} S])^2 = 4 (\phi^{\dagger} \phi)^2$, $\mathrm{Tr}[S^{\dagger} S S^{\dagger} S]= 2 (\phi^{\dagger} \phi)^2$.
In conclusion, the most general quartic potential for one single scalar field $\phi$ is given by:

\begin{equation}
V = \lambda (\mathrm{Tr} [S^{\dagger} S])^2  \, .
\end{equation}

We now consider gauge-symmetric scalar bilinears, which may cause spontaneous flavour symmetry breaking:

\begin{equation}
\mathcal{S}_1 = \langle S^{\dagger} S \rangle \: , \qquad \mathcal{S}_2 = \langle S^T \epsilon S \rangle \: .
\end{equation}
%
Under the flavour transformation $S \to SM$, $M \in$ Sp(2), they transform as:

\begin{equation}
\mathcal{S}_1 \to M^{\dagger} \mathcal{S}_1 M \: , \qquad \mathcal{S}_2 \to M^T \mathcal{S}_2 M \: .
\end{equation}
%
In the following, we show that both $\mathcal{S}_1$ and $\mathcal{S}_2$ are symmetric under this transformation, and therefore there cannot be spontaneous flavour symmetry breaking in a theory with a singe scalar field. In the case of $\mathcal{S}_1$, this follows directly from equation \ref{app_S1}:

\begin{equation}
\mathcal{S}_1 = \langle \phi^{\dagger} \phi \rangle \id \to M^{\dagger} \mathcal{S}_1 M = \langle \phi^{\dagger} \phi \rangle  M^{\dagger} M = \mathcal{S}_1 \: .
\end{equation}
%
We rewrite $\mathcal{S}_2$ as:

\begin{equation}
\begin{split}
& (S^T \epsilon S)_{11}=  \epsilon_{ij} \phi_i \phi_i = 0 \\
& (S^T \epsilon S)_{22} = -\epsilon_{ij} \phi^*_i \phi^*_j= 0 \\
& (S^T \epsilon S)_{12} = -(S^{\dagger} \epsilon S)_{21} = \delta_{ij} \phi_i \phi^*_j = \phi^{\dagger} \phi\\
\end{split}
\quad \Rightarrow  \mathcal{S}_2 = \langle \phi^{\dagger} \phi \rangle\, E \: ,
\label{app_S2}
\end{equation}
%
implying that also $\mathcal{S}_2$ is symmetric under flavour transformations:

\begin{equation}
\mathcal{S}_2 = \langle \phi^{\dagger} \phi \rangle E \to M^T \mathcal{S}_2 M = \langle \phi^{\dagger} \phi \rangle  M^T E M = \mathcal{S}_2 \: .
\end{equation}

\subsection{Multiple scalar fields}

In the case of $N_S$ scalar fields $\phi_1$, $\phi_2$, $\dots \, \phi_{N_S}$ we define the $2 \times 2N_S$ matrix $S$ as:

\begin{equation}
S = (\phi_1, \tilde{\phi}_1, \dots, \phi_{N_S}, \tilde{\phi}_{N_S}) \, .
\end{equation}
%
A gauge transformation is expressed as:
\begin{equation}
S \to S'=US \, , \quad \mathrm{with}  \quad U \in \mathrm{SU(2)} \, .
\end{equation}
%
$S$ and $S^*$ are related to each other by: 
\begin{equation}
S^* = - \epsilon S E
\label{constraintN}
\end{equation}
%
where, in this case:
\begin{equation}
E = \mathrm{diag}(\epsilon, \dots, \epsilon) \, .
\end{equation}
%
The kinetic term is expressed as: $\mathcal{L}_{kin} = \frac{1}{2} \mathrm{Tr} \bigl[ (D_{\mu}S)^{\dagger} (D^{\mu}S) \bigr]$.
The constraint \ref{constraintN} restricts flavour transformation to be: $S \to S' = SM$, with $M \in$ Sp(2$N_S$).
Given these definitions, the same arguments of equations~\ref{quartic}, \ref{proof1}, \ref{proof2} apply, to show that the most general quartic potential for $N_S$ scalar fields is:

\begin{equation}
V = \lambda_1 \bigl( \mathrm{Tr} \bigl[ S^{\dagger} S \bigr] \bigr)^2 + \lambda_2 \mathrm{Tr} \bigl[ S^{\dagger} S S^{\dagger} S\bigr] \, .
\end{equation}




