\chapter{Conclusions}


In this thesis, we presented our results of two studies of non-Abelian gauge theories applied to Beyond the Standard model physics. 
The first study, based on non-perturbative lattice techniques, is part of the ongoing large effort to explore composite Higgs models. 
More specifically, our study aims at providing non-perturbative information on the dynamics of fundamental partial compositeness models, featuring fermionic and scalar matter fields.
With this motivation, we studied on the lattice the SU(2) gauge theory with two fundamental fermions and one fundamental scalar. Given the very large parameter space of this lattice theory, our analysis is still preliminary, but nevertheless it already provides some interesting results. 
We have found evidence of a phase with broken fermion flavour symmetry, which seems to confirm the existence of the scenario postulated in fundamental partial compositeness models. 
Moreover, we found a very impressive agreement between chiral extrapolations of the meson spectrum obtained in our model and in the SU(2) lattice theory with two fundamental fermions and no scalar. 
We also addressed the question of gauge symmetry breaking induced by the scalar field, and discussed whether regions of broken and unbroken symmetry can be uniquely identified in the phase diagram. We found evidence of a symmetry-breaking transition accompanied by a thermodynamic transition affecting all the considered observables.

In the second study, we presented a perturbative computation of the transport coefficients for theories in the conformal window. 
This class of theories is very relevant in the context of Beyond the Standard Model physics, in particular for composite Higgs models. We applied perturbative results for the transport coefficients of a hot gauge theory to theories with an infrared perturbative fixed point. One interesting point of our analysis is that it provides a framework in which the perturbative results hold both at high and low temperatures, due to the perturbative nature of the infrared fixed point. 
However, it turns out that the results are quantitatively (and even just qualitatively) trustable in a very restricted range of the conformal window. 