\chapter{Conclusions}



In this thesis, we presented our results on two studies of non-Abelian gauge theories applied to Beyond the Standard model physics. 

The first study, based on non-perturbative lattice techniques, is part of the ongoing large effort to explore composite Higgs models. 
More specifically, our study aims to provide non-perturbative information on the dynamics of fundamental partial compositeness models, featuring fermionic and scalar matter fields.
With this motivation, we studied the SU(2) lattice gauge theory with two fundamental fermions and one fundamental scalar. Given the very large parameter space of this theory, our analysis is still preliminary, but nevertheless it already provides some interesting results. 
We have found evidence of a phase with broken fermion flavour symmetry, which seems to confirm the existence of the scenario postulated in fundamental partial compositeness models. 
Moreover, we found a very impressive agreement between chiral extrapolations of the meson spectrum obtained in our model and in the SU(2) lattice theory with two fundamental fermions and no scalar. 
We also addressed the question of gauge symmetry breaking induced by the scalar field, and discussed whether regions of broken and unbroken symmetry can be uniquely identified in the phase diagram. We found evidence of a symmetry-breaking transition accompanied by a thermodynamic transition affecting all the considered observables. In the future, we plan on extending our analysis to a broader range of parameter space, in order to give a detailed description of the phase structure of the model. We plan on measuring the spectrum of bound sates of the scalar field with more refined techniques than the ones used in this thesis. We will give particular attention to fermionic bound states formed by one fermion and one scalar, which are very relevant in the fundamental partial compositeness scenario. Moreover, we will have to address the renormalisation of the lattice observables, which, among other things, will provide the observable needed to set the scale and measure the physical value of the lattice spacing. Finally, we will attempt an extrapolation to small values of the lattice spacing, matching our lattice model to an effective theory with ultraviolet cutoff. We expect this extrapolation to provide valuable input for phenomenological models.


In the second study, we presented a perturbative computation of the transport coefficients of theories in the conformal window, i.e. featuring a nontrivial infrared fixed point. 
This class of theories is very relevant in the context of Beyond the Standard Model physics, in particular for composite Higgs models. One interesting point of our analysis is that it provides a framework in which the perturbative results for transport coefficients hold both at high and low temperatures, due to the perturbative nature of the infrared fixed point. 
However, it turns out that the results are quantitatively (and even just qualitatively) trustable in a very restricted range of the conformal window. It would be very interesting to extend this analysis to a broader region of the conformal window. In order to do so, the transport coefficients should be known at a higher order in perturbation theory. However, even a leading-order analysis of transport coefficients seems to be already pretty challenging.