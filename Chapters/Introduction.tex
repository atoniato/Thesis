\chapter{Introduction}

In this thesis we present a detailed study of two applications of non-Abelian gauge theories to Beyond the Standard Model physics.
The Standard Model (SM) of particle physics is very successful in describing experimental data. Nevertheless, we know that it is just an effective theory, valid only up to an ultraviolet cutoff. 
The question of what kind of new physics may constitute extensions of the Standard Model is very fascinating, and a large part of the high-energy-physics community is dedicated to uncover what lies beyond it. 
%In this thesis, we consider two deeply related topics belonging to Beyond the Standard Model research: composite Higgs models and the conformal window.

In the first study, we address the question of mass generation for SM particles. 
We consider composite Higgs models which aim to provide a more theoretically consistent framework than the Standard Model Higgs mechanism. 
In order to be phenomenologically viable, almost all such models require  a ``walking'' behaviour. These theories have an approximate infrared fixed point, and are believed to exist in theory space close to the lower boundary of the so-called conformal window, where models exhibit large-distance conformality, due to an infrared fixed point in the renormalisation group flow. 
Many lattice studies have been devoted to the search of ``walking'' theories in the last decade, which has proven to be a very difficult task.
In this thesis, we consider a novel approach for mass generation, dubbed ``fundamental partial compositeness'' \cite{Sannino:2016sfx} in which (coloured) fundamental scalar fields are introduced.

We study for the first time, the simplest such model based on the SU(2) gauge theory with two fundamental fermion flavours and one fundamental coloured scalar field. This model is intended as a minimal scenario for testing the non-perturbative dynamics of recently proposed models of  fundamental partial compositeness. These are composite Higgs models featuring fermionic and scalar matter fields, which are able to provide a complete theory of flavour, alternative to the Standard Model Higgs mechanism. This is in contrast to traditional models of partial compositeness, where, due to model building difficulties, the focus is only on the top sector.


The second study, we explore the thermodynamic properties of theories in the conformal window. In particular, we study the transport coefficients of theories in the perturbative region of the conformal window.


This thesis is organised as follows: in chapter \ref{LFT}, we give a general introduction to lattice field theory, mostly focused on the tools needed for the lattice study presented in this thesis. In chapter \ref{GFS}, after introducing composite Higgs models and the fundamental partial compositeness mechanism, we present the results of our lattice study of the SU(2) gauge theory with two fundamental fermions and one fundamental scalar. In chapter \ref{CW}, we present our study of the transport coefficients of theories in the conformal window. Before discussing our results, we provide the definitions of the transport coefficients under analysis, and we describe the method used by the authors of \cite{Arnold:2000dr} to obtain perturbative estimates of the transport coefficients of a hot gauge theory, which constitute the basis of our computations.
In appendix, some general definitions and additional material can be found.