\chapter{Introduction}

This thesis has as a common topic non-Abelian gauge theories applied to Beyond the Standard Model physics.
The Standard Model of particle physics is very successful in describing experimental data. Nevertheless, we know that it is just an effective theory, valid only up to an ultraviolet cutoff. 
The question of what kind of new physics may constitute extensions of the Standard Model is very fascinating, and it occupies a consistent part of the high-energy-physics community. 
In this thesis, we consider two deeply related topics belonging to Beyond the Standard Model research: the first is composite Higgs models and the second the conformal window.
Composite Higgs models address the question of mass generation, and aim at providing a more theoretically consistent framework with respect to the Standard Model Higgs mechanism. 
In order to build phenomenologically viable composite Higgs models, gauge theories with a "walking" coupling are needed. These theories have an approximate infrared fixed point, and are found in theory space close to the lower boundary of the so-called conformal window. Theories in the conformal window display large-distance conformality, due to an infrared fixed point in the renormalisation group flow of the gauge coupling.

In this thesis we present a lattice study of the SU(2) gauge theory with two fundamental fermion flavours and one fundamental scalar field. This model is intended as a minimal scenario for testing the non-perturbative dynamics of recently proposed models of  fundamental partial compositeness \cite{Sannino:2016sfx}. These are composite Higgs models featuring fermionic and scalar matter fields, which are able to provide a complete theory of flavour, alternative to the Standard Model Higgs mechanism. 
The second study presented in this thesis regards the thermodynamic properties of theories in the conformal window. In particular, we study the transport coefficients of theories in the perturbative conformal window.

This thesis is organised as follows: in chapter \ref{LFT}, we give a general introduction to lattice field theory, mostly focused on the tools needed for the lattice study presented in this thesis. In chapter \ref{GFS}, after introducing composite Higgs models and the fundamental partial compositeness mechanism, we present the results of our lattice study of the SU(2) gauge theory with two fundamental fermions and one fundamental scalar. In chapter \ref{CW}, we present our study of the transport coefficients of theories in the conformal window. Before discussing our results, we provide the definitions of the transport coefficients under analysis, and we describe the method used by the authors of \cite{Arnold:2000dr} to obtain perturbative estimates of the transport coefficients of a hot gauge theory, which constitute the basis of our computations.
In appendix, some general definitions and additional material can be found.