\chapter{Transport coefficients in the conformal window}

Gauge theories with a nontrivial infrared (IR) fixed point are important in the context of model building Beyond the Standard Model \cite{Sannino:2008ha}. These theories are characterised by scale invariance in the infrared and, in the case of four space-time dimensions, they are invariant under the larger group of conformal transformations. For this reason, the region in theory space, parametrised by the number of colours ($N$) and fermion flavours ($N_f$), such that a nontrivial IR fixed point exists, is called conformal window.
Substantial effort has been taken in studying the properties of theories in the conformal window, both perturbatively and via lattice simulations (\emph{references?}). In this chapter, we present a study of the transport coefficients of theories in the perturbative conformal window, which has been published in \cite{Toniato:2016twr}.
The chapter is organised as follows: we start by introducing the conformal window and the definition of the transport coefficients under analysis. We then briefly sketch the method used by the authors of \cite{Arnold:2000dr} for calculating the transport coefficients of a gauge theory coupled to fermions. Finally, we present the results obtained by applying the perturbative results of \cite{Arnold:2000dr} to theories in the conformal window. 

%%%%%%%%%%%%%%%%%%%%%%%%%%%%%%%%%%%%%%%%%%%%%%%%%%%%%%

\section{The conformal window}
\label{ conformal_window}

We consider a non-Abelian gauge theory, with gauge group SU($N$) and $N_f$ fermions in the representation $r$. The two-loop beta function for the gauge coupling $g$ is given by:

 \begin{equation}
 \beta (g) = - \frac{\beta_0}{(4\pi)^2} g^3 - \frac{\beta_1}{(4\pi)^4} g^5 + \mathcal{O}(g^{7}) \; ,
 \label{beta_f}
 \end{equation}
%
with

 \begin{align}
\beta_0 &= \frac{11}{3} C_2[G] - \frac{4}{3} T[r] N_f\; , \\
\beta_1 &= \frac{34}{3} C_2^2[G] - \left ( \frac{20}{3} C_2[G] + 4 C_2[r] \right ) T[r] N_f\; ,
\end{align}
%
where $G$ denotes the adjoint representation. The definition of the group factors $C_2$, $T$ and $d$ can be found in appendix \ref{SUN_generators}. If $\beta_0 > 0$, the theory is asymptotically free, i.e. $g=0$ is an ultraviolet(UV)-stable fixed point of the renormalisation group (RG) flow. This happens provided the number of fermions is smaller than the upper limit:

\begin{equation}
N_f^{AF} = \frac{11}{4} \frac{C_2[G]}{T[r]} \: .
\end{equation}
%
The number of flavours and colours can be chosen such that $\beta_0>0$ and $\beta_1<0$. In this case there exists an additional zero of the beta function:

\begin{equation}
g^* = -(4 \pi)^2 \frac{\beta_0}{\beta_1} \: .
\end{equation}
%
This is an IR-stable fixed point, which can be studied in perturbation theory in case $g^*$ is small \cite{Banks:1981nn}. In particular, $g^*$ goes to zero in the limit $N_f \to N_f^{AF}$. If $g^*$ is small, the theory remains perturbative along the whole RG flow, from the UV, where it is dominated by the asymptotically free fixed point $g=0$, down to the IR. If $g^*$ is not small enough, then one has to consider higher-order terms in the beta function \ref{beta_f}, and eventually, when $g^*$ leaves the perturbative regime, apply non-perturbative methods such as lattice simulations. In theory space, parametrised by $N$ and $N_f$, the upper boundary of the conformal window for every fixed $N$ is given by $N_f = N_f^{AF}$, while the lower boundary is yet to be found via non-perturbative methods, since the value of $g^*$ increases with decreasing $N_f$.


\begin{figure}[h!]
\includegraphics[width=0.5\textwidth]{pics/beta_walking}~~~
\includegraphics[width=0.5\textwidth]{pics/g_walking}
\caption{Beta function (left panel) and gauge coupling (right panel) for three different kinds of theories: a QCD-like theory (red line), a theory in the conformal window (blue line) and a theory with walking gauge coupling (green dashed line). In the right panel, the gauge coupling is plotted as a function of the energy scale $\mu$ over some reference scale $\mu_0$.} 
\label{walking}
\end{figure}


Figure \ref{walking} shows three different regimes of the beta function, and the related running of the gauge coupling. The first one (red line) characterises a QCD-like theory. The beta function has one single zero at $g=0$, and is negative for all other values of $g$. In this case, the coupling goes to zero in the UV (asymptotic freedom) and grows when moving towards the IR, until it diverges at some finite energy scale. This theory displays spontaneous chiral symmetry breaking in the IR. The second regime (blue line) characterises a theory in the conformal window. The beta function has two zeros, $g=0$ and $g=g^*$, and is negative in between. The gauge coupling goes to zero in the UV, and stabilises at the fixed point value $g=g^*$ in the IR. This theory is characterised by scale invariance in the IR, and does not display spontaneous chiral symmetry breaking. The third regime (green dashed line) is associated to "walking" dynamics. It is reasonable to assume that, close to the lower boundary of the conformal window, theories with an approximate IR fixed point can be found. Their beta function is very similar to the one of theories in the conformal window, with the difference that it gets very close to zero at $g\simeq g^*$, without actually reaching a fixed point, and then assumes a QCD-like behaviour for larger values of $g$. In this theory the gauge coupling goes to zero in the UV, then stabilises at an approximately constant value for a wide range of energies, and finally diverges at a finite energy scale, thus triggering chiral symmetry breaking. Theories with walking gauge coupling are very important in the context of composite Higgs models, as discussed in section \ref{partial_comp}. In fact these theories show an approximate conformal behaviour, which can lead to a nice separation of scales and good scaling properties of composite operators, and at the same time break chiral symmetry, which is a necessary ingredient in composite Higgs models.
Lattice simulations can be used in order to find out whether a theory is in one of the three regimes shown in figure \ref{walking}, see for example \cite{Hietanen:2008mr,Hansen:2017ejh}.

Here we consider theories in the perturbative conformal window, i.e. $N_f \lesssim N_f^{AF}$, and we study their viscosity and fermion number diffusion coefficient by applying the perturbative results for transport coefficients obtained in \cite{Arnold:2000dr}.

%REMEMBER TO JUSTIFY THE FACT THAT THE NUMBER OF FLAVOURS IS TREATED LIKE A CONTINUOUS PARAMETER!

%%%%%%%%%%%%%%%%%%%%%%%%%%%%%%%%%%%%%%%%%%%%%%%%%%%%%%

\section{Transport coefficients}


%%%%%%%%%%%%%%%%%%%%%%%%%%%%%%%%%%%%%%%%%%%%%%%%%%%%%%

\section{Transport coefficients of a gauge theory: the kinetic approach}


%%%%%%%%%%%%%%%%%%%%%%%%%%%%%%%%%%%%%%%%%%%%%%%%%%%%%%

\section{Application to theories in the conformal window}

%%%%%%%%%%%%%%%%%%%%%%%%%%%%%%%%%%%%%%%%%%%%%%%%%%%%%%
