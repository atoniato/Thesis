\chapter{Transport coefficients in the conformal window}

Gauge theories with a nontrivial infrared (IR) fixed point are important in the context of model building Beyond the Standard Model (BSM) \cite{Sannino:2008ha}. These theories are characterised by scale invariance in the infrared and, in the case of four space-time dimensions, they are invariant under the larger group of conformal transformations. For this reason, the region in theory space, parametrised by the number of colours ($N$) and fermion flavours ($N_f$), such that a nontrivial IR fixed point exists, is called conformal window.
Substantial effort has been taken in studying the properties of theories in the conformal window, both perturbatively and via lattice simulations (\emph{references?}). In this chapter, we present a study of the transport coefficients of theories in the perturbative conformal window, which has been published in \cite{Toniato:2016twr}.
The chapter is organised as follows: we start by introducing the conformal window and the definition of the transport coefficients under analysis. We then briefly sketch the method used by the authors of \cite{Arnold:2000dr} for calculating the transport coefficients of a gauge theory coupled to fermions. Finally, we present the results obtained by applying the perturbative results of \cite{Arnold:2000dr} to theories in the conformal window. 

%%%%%%%%%%%%%%%%%%%%%%%%%%%%%%%%%%%%%%%%%%%%%%%%%%%%%%

\section{The conformal window}
\label{ conformal_window}

We consider a non-Abelian gauge theory, with gauge group SU($N$) and $N_f$ fermions in the representation $r$. The two-loop beta function for the gauge coupling $g$ is given by:

 \begin{equation}
 \beta (g) = - \frac{\beta_0}{(4\pi)^2} g^3 - \frac{\beta_1}{(4\pi)^4} g^5 + \mathcal{O}(g^{7}) \; ,
 \end{equation}
%
with

 \begin{align}
\beta_0 &= \frac{11}{3} C_2[G] - \frac{4}{3} T[r] N_f\; , \\
\beta_1 &= \frac{34}{3} C_2^2[G] - \left ( \frac{20}{3} C_2[G] + 4 C_2[r] \right ) T[r] N_f\; ,
\end{align}
%
where $G$ denotes the adjoint representation. The definition of the group factors $C_2$, $T$ and $d$ can be found in appendix \ref{SUN_generators}. If $\beta_0 > 0$, the theory is asymptotically free, i.e $g=0$ is an ultraviolet(UV)-stable fixed point. This happens provided the number of fermions in smaller than the upper limit:

\begin{equation}
N_f^{AF} = \frac{11}{4} \frac{C_2[G]}{T[r]} \: .
\end{equation}

%REMEMBER TO JUSTIFY THE FACT THAT THE NUMBER OF FLAVOURS IS TREATED LIKE A CONTINUOUS PARAMETER!

%%%%%%%%%%%%%%%%%%%%%%%%%%%%%%%%%%%%%%%%%%%%%%%%%%%%%%

\section{Transport coefficients}


%%%%%%%%%%%%%%%%%%%%%%%%%%%%%%%%%%%%%%%%%%%%%%%%%%%%%%

\section{Transport coefficients of a gauge theory: the kinetic approach}


%%%%%%%%%%%%%%%%%%%%%%%%%%%%%%%%%%%%%%%%%%%%%%%%%%%%%%

\section{Application to theories in the conformal window}

%%%%%%%%%%%%%%%%%%%%%%%%%%%%%%%%%%%%%%%%%%%%%%%%%%%%%%
